% ----------------------------------------------------------
% Introdução (exemplo de capítulo sem numeração, mas presente no Sumário)
% ----------------------------------------------------------
\chapter*[Introdução]{Introdução}
	\addcontentsline{toc}{chapter}{Introdução}
	
	O \textit{MATLAB} (\textit{MATrix LABoratory}) é um \textit{software} de alta performance, capaz de lidar com matrizes, processamento de sinais e conta com inúmeras ferramentas para a construção de gráficos, entrada e saída de arquivos, como planilhas \textit{.csv}, arquivos de áudio \textit{.wav}, imagens \textit{.png}, dentre outros.
	
	Neste trabalho é apresentado a solução para uma série de atividades propostas pelo Professor Fernando Pacheco com a finalidade de exercitar e aprender mais sobre áudio, processamento de sinais e \textit{MATLAB}.
	
	\section[Objetivos]{Objetivos}
	
	A partir das atividades propostas, compreende-se que são objetivos deste trabalho criar funções para:
	
	\begin{itemize}
		\item geração de um sinal senoidal;
		\item geração de tons \textit{DTMF} (\textit{Dual-Tone Multi-Frequency});
		\item re-amostrar pela metade da frequência;
		\item re-amostrar pelo dobro da frequência;
		\item geração de um sinal de eco;
		\item aplicação de uma resposta de impulso (\textit{IR}) extraída de diversos ambientes em um sinal de voz gravada utilizando a técnica da convolução no tempo.
	\end{itemize}
	
	Para cada uma destas funções, serão realizados testes e seus resultados, comentados.