%% abtex2-modelo-relatorio-tecnico.tex, v-1.9.6 laurocesar
%% Copyright 2012-2016 by abnTeX2 group at http://www.abntex.net.br/ 
%%
%% This work may be distributed and/or modified under the
%% conditions of the LaTeX Project Public License, either version 1.3
%% of this license or (at your option) any later version.
%% The latest version of this license is in
%%   http://www.latex-project.org/lppl.txt
%% and version 1.3 or later is part of all distributions of LaTeX
%% version 2005/12/01 or later.
%%
%% This work has the LPPL maintenance status `maintained'.
%% 
%% The Current Maintainer of this work is the abnTeX2 team, led
%% by Lauro César Araujo. Further information are available on 
%% http://www.abntex.net.br/
%%
%% This work consists of the files abntex2-modelo-relatorio-tecnico.tex,
%% abntex2-modelo-include-comandos and abntex2-modelo-references.bib
%%

% ------------------------------------------------------------------------
% ------------------------------------------------------------------------
% abnTeX2: Modelo de Relatório Técnico/Acadêmico em conformidade com 
% ABNT NBR 10719:2015 Informação e documentação - Relatório técnico e/ou
% científico - Apresentação
% ------------------------------------------------------------------------ 
% ------------------------------------------------------------------------

\documentclass[
	% -- opções da classe memoir --
	12pt,				% tamanho da fonte
	article,			% não quebra páginas em novos capitulos
	openright,			% capítulos começam em pág ímpar (insere página vazia caso preciso)
	%twoside,			% para impressão em recto e verso. Oposto a oneside
	oneside,
	a4paper,			% tamanho do papel. 
	% -- opções da classe abntex2 --
	chapter=TITLE,		% títulos de capítulos convertidos em letras maiúsculas
	section=TITLE,		% títulos de seções convertidos em letras maiúsculas
	%subsection=TITLE,	% títulos de subseções convertidos em letras maiúsculas
	%subsubsection=TITLE,% títulos de subsubseções convertidos em letras maiúsculas
	% -- opções do pacote babel --
	english,			% idioma adicional para hifenização
	french,				% idioma adicional para hifenização
	spanish,			% idioma adicional para hifenização
	brazil,				% o último idioma é o principal do documento
]{abntex2}

\usepackage{style}

% ---
% Informações de dados para CAPA e FOLHA DE ROSTO
% ---
\titulo{FLYBACK COM FAN7554}
\autor{João Antônio Cardoso \and Jonas Vieira de Souza}
\local{Florianópolis}
\data{\today}
%\data{2015, v-1.9.6}
\orientador[Prof. ]{Joabel Moia}
\instituicao{Instituto Federal de Educação, Ciência e Tecnologia de Santa Catarina - Campus Florianópolis \\
	Engenharia Eletrônica \\
	Eletrônica de Potência II
}
\tipotrabalho{Relatório técnico}
% O preambulo deve conter o tipo do trabalho, o objetivo, 
% o nome da instituição e a área de concentração 
\preambulo{\@title - relatório técnico para aprovação na parte da disciplina de Eletrônica de Potência II.}
% ---

% ---
% Modificando a capa padrão
% ---
\renewcommand{\imprimircapa}{
	\begin{capa}
		\SingleSpacing
		\begin{adjustwidth}{}{}			
			\begin{minipage}{1\textwidth}
				\begin{wrapfigure}{l}{0.25\textwidth}
					\vspace*{-1em}\includegraphics[height=4em]{logotipo_ifsc.jpg}
				\end{wrapfigure}
				\imprimirinstituicao 
			\end{minipage}
			\vfill
			\begin{center}\ABNTEXchapterfont\LARGE\imprimirtitulo
				\vfill
				\small
				\begin{center}
					Aluno: \imprimirautor\\[1em]
					\imprimirorientadorRotulo\imprimirorientador\\
					\imprimircoorientadorRotulo\imprimircoorientador
				\end{center}
				\vfill
				\imprimirlocal, \imprimirdata
			\end{center}	
		\end{adjustwidth}
	\end{capa}
}


% ---
% compila o indice
% ---
\makeindex
% ---

% ----
% Início do documento
% ----
\begin{document}
	
	% Seleciona o idioma do documento (conforme pacotes do babel)
	%\selectlanguage{english}
	\selectlanguage{brazil}
	
	% Retira espaço extra obsoleto entre as frases.
	\frenchspacing 
	
	% ----------------------------------------------------------
	% ELEMENTOS PRÉ-TEXTUAIS
	% ----------------------------------------------------------
	% \pretextual
	
	% ---
	% Capa
	% ---
	\imprimircapa
	% ---
	
	% ---
	% inserir o sumario
	% ---
	\pdfbookmark[0]{\contentsname}{toc}
	\tableofcontents*
	%\cleardoublepage
	% ---
	
	% ----------------------------------------------------------
	% ELEMENTOS TEXTUAIS
	% ----------------------------------------------------------
	\textual
	
	\pagebreak
	
	% ----------------------------------------------------------
% Introdução (exemplo de capítulo sem numeração, mas presente no Sumário)
% ----------------------------------------------------------
\chapter*[Introdução]{Introdução}
	\addcontentsline{toc}{chapter}{Introdução}
	
	O \textit{MATLAB} (\textit{MATrix LABoratory}) é um \textit{software} de alta performance, capaz de lidar com matrizes, processamento de sinais e conta com inúmeras ferramentas para a construção de gráficos, entrada e saída de arquivos, como planilhas \textit{.csv}, arquivos de áudio \textit{.wav}, imagens \textit{.png}, dentre outros.
	
	Neste trabalho é apresentado a solução para uma série de atividades propostas pelo Professor Fernando Pacheco com a finalidade de exercitar e aprender mais sobre áudio, processamento de sinais e \textit{MATLAB}.
	
	\section[Objetivos]{Objetivos}
	
	A partir das atividades propostas, compreende-se que são objetivos deste trabalho criar funções para:
	
	\begin{itemize}
		\item geração de um sinal senoidal;
		\item geração de tons \textit{DTMF} (\textit{Dual-Tone Multi-Frequency});
		\item re-amostrar pela metade da frequência;
		\item re-amostrar pelo dobro da frequência;
		\item geração de um sinal de eco;
		\item aplicação de uma resposta de impulso (\textit{IR}) extraída de diversos ambientes em um sinal de voz gravada utilizando a técnica da convolução no tempo.
	\end{itemize}
	
	Para cada uma destas funções, serão realizados testes e seus resultados, comentados.
	\section[Desenvolvimento]{Desenvolvimento}
	
	\subsection{Gerando um tom}
	
	Com a finalidade de iniciar os estudos com áudio, a primeira função a ser criada é gerar um tom senoidal e verificar sua funcionalidade no tempo e na frequência, através da transformada rápida de \textit{Fourier} (\textit{FFT}). A função criada consta no \autoref{tonegen.m}, e sua utilização, no \autoref{questao1.m}, no qual foi gerada uma senoide em $440 [Hz]$, de amplitude $0,1$, de $100$ segundos de duração, amostrado em $8000\,[Hz]$ e em seguida, plotados no tempo (apenas um período), e na frequência. Os resultados podem ser verificados na \autoref{fq1}.
	
	\begin{equation}
	\label{eq-eco1}
	s_e(t)=\alpha\,s(t -T)
	\end{equation}
	
	\begin{equation}
	\label{eq-eco2}
	r(t)=s(t)+s_e(t)
	\end{equation}
	
	O sinal de eco facilmente perceptível no áudio, e como é composto apenas das frequências do sinal original, pouco é alterado em sua resposta na frequência (\autoref{fq8-fft}). No entanto, na \autoref{fq8-time} é perceptível que o sinal ficou mais denso - as regiões onde apresentavam um silêncio, passaram a ser preenchidas pelo eco, embora ele não seja longo o suficiente para gerar uma calda muito extensa, no final do arquivo, embora já seja perceptível.
	
	%\begin{figure}[htbp]
	%	\centering
	%	\caption{Comparativo de tempo de processamento em relação à duração do \textit{IR}}
	%	\includegraphics[width=\textwidth,height=240px,keepaspectratio]{imgs/q9_convs.png}
	%	\label{fq9-convs}
	%	\legend{Comparativo de tempo de processamento em relação à duração do \textit{IR}. Fonte: do autor. }
	%\end{figure}
		
	
	Para compreender quanto tempo leva o processamento, fez-se à uma regressão linear (usando o próprio \textit{MATLAB}), o qual pode ser corrigida para aproximadamente $t_{processamento} \approx 1,28*T$, onde $T$ é a duração do \textit{IR}, e vale exclusivamente para o arquivo de áudio gravado, executado neste específico computador. Com tal relação, podemos notar principalmente, que o tempo de processamento será maior que o tempo de duração do \textit{IR}, portanto, inviável para tempo real.
	%\input{texto2.tex}
		
	% ---
	% Finaliza a parte no bookmark do PDF
	% para que se inicie o bookmark na raiz
	% e adiciona espaço de parte no Sumário
	% ---
	\phantompart
	
	% ---
% Conclusão
% ---
\section[Conclusões e recomendações]{Conclusões e recomendações}
% ---

    Este trabalho apresentou de modo resumido a implementação de uma série de funções para processamento de áudio no \textit{MATLAB}, assim como comentários sobre os resultados obtidos, que sempre quando possível, foram relacionados com fenômenos da música.
    
    Foi perceptível que uma ferramenta como \textit{MATLAB} permite o estudo de processamento de sinal para áudio, executando \textit{scripts} programados pelo usuário, abrindo, gerando e gravando arquivos em disco que podem ser utilizados por outros \textit{softwares}.
    
    Por fim, notou-se que há uma limitação quanto ao poder de processamento no que diz respeito à técnica utilizada para realizar a convolução, portanto, fica o anseio por estudar técnicas que permitam realizar a convolução em tempo real.
	
	% ----------------------------------------------------------
	% ELEMENTOS PÓS-TEXTUAIS
	% ----------------------------------------------------------
	\postextual
	
	% ----------------------------------------------------------
	% Referências bibliográficas
	% ----------------------------------------------------------
	%\pagebreak
	%\bibliography{referencias}
	
	% ----------------------------------------------------------
	% Glossário
	% ----------------------------------------------------------
	%
	% Consulte o manual da classe abntex2 para orientações sobre o glossário.
	%
	%\glossary
	
	% ----------------------------------------------------------
	% Apêndices
	% ----------------------------------------------------------
	
	% ---
	% Inicia os apêndices
	% ---
	%\begin{apendicesenv}
	%	
	%	% Imprime uma página indicando o início dos apêndices
	%	\partapendices
	%	
	%	\chapter{questao1.m}
	%	\label{questao1.m}
	%	\lstinputlisting{questao1.m}
	%	\pagebreak
	%	
	%\end{apendicesenv}
	% ---
	
	% ----------------------------------------------------------
	% Anexos
	% ----------------------------------------------------------
	
	% ---
	% Inicia os anexos
	% ---
%	\begin{anexosenv}
%		
%		% Imprime uma página indicando o início dos anexos
%		\partanexos
%		
%	\end{anexosenv}
	
	%---------------------------------------------------------------------
	% INDICE REMISSIVO
	%---------------------------------------------------------------------
	
	\phantompart
	
	\printindex
		
\end{document}

