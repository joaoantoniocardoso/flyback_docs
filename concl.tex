% ---
% Conclusão
% ---
\section[Conclusões e recomendações]{Conclusões e recomendações}
% ---

    Este trabalho apresentou de modo resumido a implementação de uma série de funções para processamento de áudio no \textit{MATLAB}, assim como comentários sobre os resultados obtidos, que sempre quando possível, foram relacionados com fenômenos da música.
    
    Foi perceptível que uma ferramenta como \textit{MATLAB} permite o estudo de processamento de sinal para áudio, executando \textit{scripts} programados pelo usuário, abrindo, gerando e gravando arquivos em disco que podem ser utilizados por outros \textit{softwares}.
    
    Por fim, notou-se que há uma limitação quanto ao poder de processamento no que diz respeito à técnica utilizada para realizar a convolução, portanto, fica o anseio por estudar técnicas que permitam realizar a convolução em tempo real.